
\subsection{Literature Survey}
Thermal image analysis has evolved from simple thresholding methods to sophisticated deep learning models. Early systems relied on manual inspection or hardware-intensive sensors to estimate material heat exposure, offering limited scalability and consistency \cite{Li2023}. Convolutional Neural Networks (CNNs) later enabled automated feature extraction from infrared images, but often lacked the capacity to capture both local textures and global context effectively \cite{Zheng2021}. Vision Transformers (ViTs) have recently emerged as a powerful alternative, leveraging patch-based self-attention to model complex spatial relationships in images \cite{Dosovitskiy2021,Liu2024}.

\subsubsection{Related Work}

\paragraph{Thermal Image Classification}
\begin{itemize}[leftmargin=*]
    \item \textbf{Thresholding Techniques:} Rule-based segmentation of temperature ranges, prone to noise and manual tuning \cite{Li2023}.
    \item \textbf{CNN-Based Models:} Use convolutional layers to learn spatial features from thermal images, improving automation but missing global context, leading to misclassifications in complex scenes \cite{Zheng2021}.
    \item \textbf{Vision Transformer Adaptation:} Recent works fine-tune ViT models—originally trained on large RGB datasets—for infrared data. Patch embedding (e.g., 8×8) and positional encoding enable modeling of both local thermal textures and global patterns \cite{Dosovitskiy2021,Liu2024}.  
\end{itemize}

\paragraph{Infrared Camera Calibration and Emissivity Correction}
\begin{itemize}[leftmargin=*]
    \item \textbf{Emissivity Correction:} Correcting thermal data using material-specific emissivity tables ensures accurate temperature readings. Neural networks have been applied to refine emissivity correction and improve thermogram quality \cite{Schmid2025}.
    \item \textbf{Pulsed Thermography Reconstruction:} Techniques in Eddy Current Pulsed Thermography combine emissivity correction with pattern reconstruction to enhance defect visibility \cite{Li2023}.
\end{itemize}

\paragraph{Multi-Class Classification in Industrial Contexts}
\begin{itemize}[leftmargin=*]
    \item \textbf{Construction Material Classification:} Multi-class classifiers label materials into discrete temperature categories (e.g., 200 °C, 400 °C, 600 °C, 800 °C), aiding in safety monitoring \cite{Li2024}.
    \item \textbf{Semiconductor Wafer Map Failure Recognition:} Deep CNNs classify wafer defects into multiple failure modes, highlighting the importance of fine-grained classification pipelines \cite{Zheng2021}.
\end{itemize}

\paragraph{Real-Time Deployment Techniques}
\begin{itemize}[leftmargin=*]
    \item \textbf{Quantization and Integer-Only Inference:} Quantization reduces model size and latency, enabling integer-arithmetic-only inference for edge devices \cite{Jacob2018}.
    \item \textbf{Edge AI Optimization:} Deployment strategies using TorchScript and TensorRT optimize deep models for real-time performance on GPUs and CPUs \cite{Staecker2021}.
    \item \textbf{Mixed-Precision Training:} NVIDIA Apex enables FP16/FP32 mixed-precision, balancing speed and accuracy \cite{Staecker2021}.
\end{itemize}

\paragraph{Explainable AI for Thermal Models}
\begin{itemize}[leftmargin=*]
    \item \textbf{Grad-CAM:} Provides visual explanations by highlighting image regions influencing model decisions. Adaptable to ViT architectures for thermal data \cite{Selvaraju2020}.
    \item \textbf{Grad-CAM++:} An extension improving localization by generalizing gradient-based attribution, beneficial for fine-grained thermal defect detection \cite{Chattopadhyay2018}.
\end{itemize}

\subsubsection{Research Gaps}
Despite progress, several gaps remain:
\begin{itemize}[leftmargin=*]
    \item \textbf{Lack of Public Multi-Class Thermal Datasets:} Standardized, labeled datasets covering a broad temperature range for construction materials are scarce, hindering reproducibility \cite{Li2023}.
    \item \textbf{ViT Adaptation Challenges:} Pretrained ViTs on RGB require careful fine-tuning for infrared inputs, including novel augmentation and positional encoding techniques \cite{Dosovitskiy2021,Liu2024}.
    \item \textbf{Real-Time, User-Friendly Interfaces:} Existing tools often demand technical expertise or specialized hardware, lacking simple GUIs for field deployment \cite{Liu2024}.
    \item \textbf{Explainability in Thermal Contexts:} Grad-CAM methods need adaptation to capture subtle thermal anomalies across materials \cite{Selvaraju2020,Chattopadhyay2018}.
    \item \textbf{Edge Deployment Reliability:} Ensuring sub-second inference on CPU-only devices remains challenging without extensive optimization \cite{Staecker2021}.
\end{itemize}

\subsection{Software Requirement Specification}

\subsubsection{Introduction}

\paragraph{Purpose}
The purpose of this document is to outline all software requirements for \textbf{ThermoSight}, an AI-driven system that classifies construction materials into four heat-exposure categories (200 °C, 400 °C, 600 °C, 800 °C) using thermal images. It serves as the authoritative reference for developers, testers, and project managers to ensure the final product meets user needs and design objectives.

\paragraph{Intended Audience and Reading Suggestions}
\begin{itemize}[leftmargin=*]
    \item \textbf{Project Stakeholders:} Understand project goals, deliverables, and acceptance criteria.
    \item \textbf{Software Developers and Data Scientists:} Implement the ViT model, data pipelines, and user interface.
    \item \textbf{Project Managers:} Plan resources, track progress, and verify compliance with requirements.
    \item \textbf{Quality Assurance Engineers:} Design test cases aligned with functional and performance specifications.
\end{itemize}
Readers should be familiar with:
\begin{itemize}[leftmargin=*]
    \item Thermal imaging fundamentals and infrared data formats.
    \item Vision Transformer architectures and deep learning workflows.
    \item GUI development for image processing applications.
\end{itemize}

\paragraph{Project Scope}
ThermoSight focuses on fine-grained classification of construction materials under varying heat exposures. The system includes:
\begin{itemize}[leftmargin=*]
    \item A preprocessing pipeline: Emissivity correction, filtering, cropping, and augmentation.
    \item A Vision Transformer backend: 12-layer ViT-Base fine-tuned for infrared inputs.
    \item An explainability module: Grad-CAM overlays highlighting thermal anomalies.
    \item A user interface: Streamlit-based desktop/web app for image upload, inference, and visualization.
    \item Real-time deployment: TorchScript serialization and TensorRT quantization for edge inference.
\end{itemize}

\subsubsection{Overall Description}

\paragraph{Product Perspective}
ThermoSight builds upon advances in computer vision and thermal imaging to deliver a robust software solution for industrial safety monitoring. Unlike legacy systems relying on fixed thresholds, ThermoSight’s ViT core learns complex local textures and global thermal patterns directly from infrared images \cite{Dosovitskiy2021}. The product can operate as a standalone desktop application or integrate into larger safety monitoring platforms via RESTful APIs \cite{Staecker2021}.

\paragraph{Product Features}
\begin{itemize}[leftmargin=*]
    \item \textbf{Fine-Grained Thermal Classification:} Automatically assigns materials to one of four precise heat-exposure categories (200 °C, 400 °C, 600 °C, 800 °C) with ≥94\% accuracy \cite{Liu2024,Li2024}.
    \item \textbf{Emissivity Correction:} Applies material-specific emissivity tables and neural network–based refinement for accurate temperature mapping \cite{Schmid2025,Li2023}.
    \item \textbf{Vision Transformer Backbone:} Employs a 12-layer ViT with 8×8 patch embedding and multi-head self-attention to capture both local thermal textures and global patterns \cite{Dosovitskiy2021,Liu2024}.
    \item \textbf{Explainable Heatmaps:} Grad-CAM overlays highlight critical thermal regions, enhancing inspector trust and meeting regulatory transparency requirements \cite{Selvaraju2020,Chattopadhyay2018}.
    \item \textbf{Real-Time Inference:} Processes 460×460 px thermal images in under 350 ms on GPU; CPU fallback <3 s via TorchScript and TensorRT INT8 quantization \cite{Jacob2018,Staecker2021}.
    \item \textbf{User-Friendly Interface:} Streamlit desktop/web app with intuitive controls for loading model checkpoints, uploading images, and visualizing results.
    \item \textbf{Modular, Edge-Ready Design:} Decoupled preprocessing, model inference, and UI modules enable batch processing, Docker deployment, and edge integration on devices like NVIDIA Jetson \cite{Staecker2021}.
\end{itemize}

\subsubsection{External Interface Requirements}

\paragraph{User Interfaces}
\begin{itemize}[leftmargin=*]
    \item \textbf{Image Upload Panel:} Drag-and-drop or file-browse to load single or multiple thermal images (PNG, JPEG, radiometric TIFF).
    \item \textbf{Model Management:} Dropdown to select ViT checkpoint versions (e.g., Date/Accuracy).
    \item \textbf{Inference Dashboard:} Displays predicted temperature class, confidence score, and Grad-CAM heatmap overlay.
    \item \textbf{Settings Dialog:} Adjust preprocessing parameters (patch size, normalization) and output formats (CSV, JSON, PDF).
\end{itemize}

\paragraph{Hardware Interfaces}
\begin{itemize}[leftmargin=*]
    \item \textbf{Infrared Camera Integration (Optional):} USB/SDK support for FLIR and Seek Thermal cameras to capture images directly.
    \item \textbf{Workstation/Edge Device:} PC with optional NVIDIA GPU (CUDA-enabled) or CPU-only fallback. Jetson support ensures field portability.
    \item \textbf{Thermocouple Logging (Offline):} External K-type thermocouples to record ground-truth surface temperatures during data collection.
\end{itemize}

\paragraph{Software Interfaces}
\begin{itemize}[leftmargin=*]
    \item \textbf{REST API Endpoints:} 
    \begin{itemize}[leftmargin=*]
        \item \texttt{/upload} – Accepts image files; returns preprocessing status.
        \item \texttt{/infer} – Accepts preprocessed image tensors; returns predicted class and confidence.
        \item \texttt{/heatmap} – Returns Grad-CAM heatmap in base64.
    \end{itemize}
    \item \textbf{Python SDK:} Functions for preprocessing, model loading, inference, and Grad-CAM generation, installable via \texttt{pip}.
    \item \textbf{Containerization:} Docker support for deploying FastAPI and Streamlit services with CUDA libraries for GPU acceleration.
    \item \textbf{Logging/Monitoring:} Prometheus metrics (P95 latency, GPU utilization) and PostgreSQL for inference logs (timestamp, file name, predicted class, confidence).
\end{itemize}

\subsubsection{Other Non-Functional Requirements}

\paragraph{Performance Requirements}
\begin{itemize}[leftmargin=*]
    \item \textbf{Latency:} Target <350 ms per 460×460 px image on NVIDIA RTX GPUs; CPU-only <3 s via INT8 quantization \cite{Staecker2021,Jacob2018}.
    \item \textbf{Throughput:} Support ≥10 concurrent inference requests with P95 latency <500 ms on mid-range GPU.
    \item \textbf{Scalability:} Modular microservices allow horizontal scaling of REST endpoints behind a load balancer.
\end{itemize}

\paragraph{Safety Requirements}
\begin{itemize}[leftmargin=*]
    \item \textbf{Data Privacy:} All image data processed locally; no external transmission. Log files anonymized and encrypted at rest.
    \item \textbf{Error Handling:} Graceful notifications for invalid file formats, missing model weights, or inference timeouts.
\end{itemize}

\paragraph{Security Requirements}
\begin{itemize}[leftmargin=*]
    \item \textbf{Access Control:} Role-based authentication for model management and inference logs. OAuth2 via Auth0 for user login.
    \item \textbf{Data Protection:} Secure storage of model checkpoints and logs; HTTPS for all REST API traffic.
    \item \textbf{Regular Audits:} Periodic vulnerability scans (e.g., Trivy) on Docker images; OWASP ZAP for API penetration testing.
\end{itemize}

\subsection{Cost Analysis}

\paragraph{Personal Costs}
Team members and faculty mentors contribute time and expertise:
\begin{itemize}[leftmargin=*]
    \item \textbf{Student Time Commitment:} Each of five members dedicates ~10–15 hours/week over 16 weeks (\(\approx\) 800–1200 total hours).
    \item \textbf{Mentor Time:} Dr.\ Rohit Ahuja provides guidance, feedback, and code reviews as part of academic duties.
\end{itemize}

\paragraph{Resource Costs}
\begin{itemize}[leftmargin=*]
    \item \textbf{Software and Tools:} 
    \begin{itemize}[leftmargin=*]
        \item PyTorch, Hugging Face Transformers, albumentations, OpenCV (open-source).
        \item Streamlit, FastAPI, Docker (open-source).
        \item Docker image storage and CI/CD minutes (GitHub Actions) – minimal cost within free tier.
    \end{itemize}
    \item \textbf{Compute and Hardware:}
    \begin{itemize}[leftmargin=*]
        \item Cloud GPU credits or on-premise NVIDIA RTX GPU (\(\approx\) \$500–\$1,000 usage).
        \item Edge device: NVIDIA Jetson Nano/Orin (\(\approx\) \$100–\$400).
    \end{itemize}
    \item \textbf{Data Acquisition:}
    \begin{itemize}[leftmargin=*]
        \item Calibration targets and materials (steel, concrete, wood, composites).
        \item Thermal camera FLIR One Pro or Seek Thermal Pro (\(\approx\) \$200–\$300).
        \item Thermocouples (K-type) and data logger (\$\approx\:50).
        \item Annotation effort: 100–150 hours of manual labeling.
    \end{itemize}
\end{itemize}

\subsection{Risk Analysis}

\subsubsection{Technical Risks}
\begin{itemize}[leftmargin=*]
    \item \textbf{Model Generalization:} ViT may underperform on unseen materials or ambient conditions without sufficient fine-tuning and data augmentation \cite{Dosovitskiy2021,Liu2024}.
    \item \textbf{Real-Time Inference:} Achieving sub-350 ms on CPU-only environments may require further optimization or hardware upgrades \cite{Staecker2021}.
    \item \textbf{Pipeline Integration:} Merging emissivity correction, ViT inference, and Grad-CAM modules could introduce compatibility issues in the Docker container.
\end{itemize}

\subsubsection{Data Risks}
\begin{itemize}[leftmargin=*]
    \item \textbf{Data Quality and Consistency:} Variations in material emissivity, camera calibration, or ambient conditions can introduce noise and misclassifications \cite{Li2023}.
    \item \textbf{Dataset Diversity:} Insufficient representation of all material types, surface finishes, and temperature ranges may limit model robustness.
\end{itemize}

\subsubsection{Project Management Risks}
\begin{itemize}[leftmargin=*]
    \item \textbf{Schedule Overruns:} Balancing ThermoSight development tasks with academic commitments may impact milestone delivery.
    \item \textbf{Team Coordination:} Distributed workflows and differing schedules necessitate clear communication and regular status updates.
\end{itemize}

\subsubsection{External Risks}
\begin{itemize}[leftmargin=*]
    \item \textbf{Environmental Variability:} Field deployments in harsh conditions (extreme weather) may compromise image quality and model accuracy.
    \item \textbf{Technological Advances:} Rapid improvements in thermal sensors or new architectures (e.g., improved ViTs) could require design updates.
    \item \textbf{Data Privacy and Compliance:} Handling site-specific thermal data could involve sensitive infrastructure, requiring compliance with data governance policies.
\end{itemize}


